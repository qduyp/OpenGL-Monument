\documentclass[12pt]{article}
\usepackage[utf8]{inputenc}
\usepackage{listings}
\usepackage{color}
\usepackage{graphicx}
\usepackage{hyperref}
\graphicspath{ {./images/} }
\usepackage[a4paper, top=2cm, bottom=2cm, right=2cm, left=2.5cm]{geometry}

\definecolor{mygreen}{rgb}{0,0.6,0}
\definecolor{mygray}{rgb}{0.5,0.5,0.5}
\definecolor{mymauve}{rgb}{0.58,0,0.82}

\lstset{ 
 % Source: https://en.m.wikibooks.org/wiki/LaTeX/Source_Code_Listings
  backgroundcolor=\color{white},   % choose the background color; you must add \usepackage{color} or \usepackage{xcolor}; should come as last argument
  basicstyle=\footnotesize,        % the size of the fonts that are used for the code
  breakatwhitespace=false,         % sets if automatic breaks should only happen at whitespace
  breaklines=true,                 % sets automatic line breaking
  captionpos=b,                    % sets the caption-position to bottom
  commentstyle=\color{mygreen},    % comment style
  deletekeywords={...},            % if you want to delete keywords from the given language
  escapeinside={\%*}{*)},          % if you want to add LaTeX within your code
  extendedchars=true,              % lets you use non-ASCII characters; for 8-bits encodings only, does not work with UTF-8
  firstnumber=1,                   % start line enumeration with line 1
  frame=single,	                   % adds a frame around the code
  keepspaces=true,                 % keeps spaces in text, useful for keeping indentation of code (possibly needs columns=flexible)
  keywordstyle=\color{blue},       % keyword style
  language=C++,                 % the language of the code
  morekeywords={*,...},            % if you want to add more keywords to the set
  numbers=left,                    % where to put the line-numbers; possible values are (none, left, right)
  numbersep=5pt,                   % how far the line-numbers are from the code
  numberstyle=\tiny\color{mygray}, % the style that is used for the line-numbers
  rulecolor=\color{black},         % if not set, the frame-color may be changed on line-breaks within not-black text (e.g. comments (green here))
  showspaces=false,                % show spaces everywhere adding particular underscores; it overrides 'showstringspaces'
  showstringspaces=false,          % underline spaces within strings only
  showtabs=false,                  % show tabs within strings adding particular underscores
  stepnumber=1,                    % the step between two line-numbers. If it's 1, each line will be numbered
  stringstyle=\color{mymauve},     % string literal style
  tabsize=4,	                   % sets default tabsize to 4 spaces
  %title=\lstname                   % show the filename of files included with \lstinputlisting; also try caption instead of title
}
\hypersetup{
    bookmarks=true,         % show bookmarks bar?
    unicode=false,          % non-Latin characters in Acrobat’s bookmarks
    pdftoolbar=true,        % show Acrobat’s toolbar?
    pdfmenubar=true,        % show Acrobat’s menu?
    pdffitwindow=false,     % window fit to page when opened
    pdfstartview={FitH},    % fits the width of the page to the window
    pdftitle={My title},    % title
    pdfauthor={Author},     % author
    pdfsubject={Subject},   % subject of the document
    pdfcreator={Creator},   % creator of the document
    pdfproducer={Producer}, % producer of the document
    pdfnewwindow=true,      % links in new PDF window
    colorlinks=true,       % false: boxed links; true: colored links
    %linkcolor=red,          % color of internal links (change box color with linkbordercolor)
    citecolor=green,        % color of links to bibliography
    filecolor=cyan,         % color of file links
    urlcolor=magenta        % color of external links
}

\author{\\ Tan Minh Ho \\ s82053 \\ \\ \\ Prof. Dr.-Ing. habil. Wolfgang Oertel \\ Computergrafik I\\ \\ }
\title{Belegarbeit - OpenGL}

\begin{document}

\maketitle 

\pagebreak

\tableofcontents

\pagebreak

\section{Aufgabenstellung}

Schreiben Sie ein Programm in C/C++, das unter Verwendung von OpenGL, Vertex- und Fragment-Shadern folgende Aufgaben realisiert. \\

\subsection{Aufgabe 1}
Geometrische Objeckte: Erzeugen Sie eine interaktive zeitlich animierte Szenze mit mehreren unterschiedlichen farblichen und texturierten dreidimensionalen geometrischen Objekten. \\
\subsection{Aufgabe 2}
Beleuchtung: Beleuchten Sie die Szenze mit verschiedenartigen Lichtquellen so, dass auf den Objekten unterschiedliche Beleuchtungseffekte sichtbar werden. \\
\subsection{Aufgabe 3}
Ansicht: Stellen Sie die Szenze gleichzeitig in verschiedenen Ansichten und Projektionen in merheren Viewports des Anzeigefensters dar. \\
\subsection{Aufgabe 4}
Programm: Stellen Sie das komplette Programm in Quelltextform als Visual-Studio-C/C++-Projekt und in ausführbarer Form als exe-File derart bereits, dass die Lauffähigkeit unter MS Windows gewährleistet ist.\\
\subsection{Aufgabe 5}
Dokumentation: Fertigen Sie eine Systemdokumentation in Form eines pdf-Dokumentes von etwa 10 Seiten an, die Deckblatt, Gliederung, Aufgabenbeschreibung, Lösungsansatz, Lösungsumsetzung, Installations- und Bedienugsanleitung,
einige Bildschirm-Snapshots, Probleme, Literatur- und Quellenverzeichis enthält. \\
\subsection{Aufgabe 6}
Abgabe: Übergeben Sie die Ergebnisse der Aufgaben 4 und 5 zusammengefasst in einem Verzeichnis "$Name_Vorname_Bibliotheksnummer$" an den Lehrenden. Bei Bedarf kann sich eine Abnahme der Belegarbeit mit Demonstration der Lauffähigkeit erforderlich machen. \\

\pagebreak

\section{Installationsanleitung}

Um die Quellcode zu compilieren, müssen OpenGL-Bibliotheken installiert werden. Sie sind 

\begin{itemize}
	\item GL
	\item GLU
	\item GLUT oder FreeGlut
	\item GLM oder FreeImage
	\item GLEW oder Glee
\end{itemize}

In Linux / MacOS kann man einfach \textbf{package manager} \hyperref[sec:pacman]{[1]} verwenden. Nach der Installation lauft das Kommando \textbf{make}. Außerdem gibt es andere Kommandos: \textbf{make clean} und \textbf{make remove}. \\
In MS Windows verwendet das Programm Visual Studio (oder eine andere IDE) zu compilieren. Und alle oberen Bibliotheken müssen vorher gebunden sein. Beachten: Das Kommando \textbf{make} funktioniert in MS Window nicht. \\

\pagebreak

\section{Lösungsumsetzung}

\subsection{Headerfile}
\lstinputlisting[firstline=1, lastline=22]{../src/libs/includes.h}

\subsection{Cube}
\lstinputlisting[firstnumber=1, lastline=91]{../src/cube.cpp}

\subsection{Pyramid}
\lstinputlisting[firstnumber=1, lastline=76]{../src/pyramid.cpp}

\subsection{Main}
\lstinputlisting[firstnumber=1, lastline=137]{../src/main.cpp}

\subsection{Limitierungen und Verbesserungsvorschläge}
\begin{itemize}
	\item In tiling-Window-Manger zeigt das Programm nur einen Teil der Szenze. Im Moment gibt es keine Lösung.
	\item Das Programm ist nicht optimiert und kann unter MS Window wegen Bibliotheken nicht laufen.
\end{itemize}

\section{Literatur und Quellenverzeichnis}
\begin{itemize}
	\item Pamcan -Rosetta : \url{wiki.archlinux.org/title/Pacman/Rosetta} \label{sec:pacman} [1]
	\item Prof. Dr. Wolfgang Oertel: Computergrafik I Vorlesungsmaterial
	\item OpenGL - Archwiki: \url{wiki.archlinux.org/title/OpenGL}
\end{itemize}
\end{document}
